% Rscript -e "library(knitr); knit('presentation.Rnw')"
% pdflatex presentation
% biber presentation
% pdflatex presentation
% pdflatex presentation
% 
% Einleitung
% 
% Überblick children kurz
% Wie fördern sie
% Über d verteilt
% Beide programme erklären
% Daten sammlung prozess
% Datenstruktur
% Datenaufbereitung
% 
% Summary Statistics
% 
% Mittagstisch
% Entdeckerfonds
% Subsidy
% Selfworth
% Daytodayskills
% Health variables
% 
% Regressionen
% 
% DiD-Ansatz
% 
% Grundlegende Idee – was wollen wir testen?
% Zielvariablen und warum
% Grafische evidenz 
% Resultate - Regressionstabellen
% 
% Fazit und Verbesserungsvorschläge
% 
% Gründe für fehlende effekte 
% Tipps zur datenerhebung und struktur
% partition


\documentclass[hyperref={pdfpagelabels=false}\usepackage[]{graphicx}\usepackage[]{color}
% maxwidth is the original width if it is less than linewidth
% otherwise use linewidth (to make sure the graphics do not exceed the margin)
\makeatletter
\def\maxwidth{ %
  \ifdim\Gin@nat@width>\linewidth
    \linewidth
  \else
    \Gin@nat@width
  \fi
}
\makeatother

\definecolor{fgcolor}{rgb}{0.345, 0.345, 0.345}
\newcommand{\hlnum}[1]{\textcolor[rgb]{0.686,0.059,0.569}{#1}}%
\newcommand{\hlstr}[1]{\textcolor[rgb]{0.192,0.494,0.8}{#1}}%
\newcommand{\hlcom}[1]{\textcolor[rgb]{0.678,0.584,0.686}{\textit{#1}}}%
\newcommand{\hlopt}[1]{\textcolor[rgb]{0,0,0}{#1}}%
\newcommand{\hlstd}[1]{\textcolor[rgb]{0.345,0.345,0.345}{#1}}%
\newcommand{\hlkwa}[1]{\textcolor[rgb]{0.161,0.373,0.58}{\textbf{#1}}}%
\newcommand{\hlkwb}[1]{\textcolor[rgb]{0.69,0.353,0.396}{#1}}%
\newcommand{\hlkwc}[1]{\textcolor[rgb]{0.333,0.667,0.333}{#1}}%
\newcommand{\hlkwd}[1]{\textcolor[rgb]{0.737,0.353,0.396}{\textbf{#1}}}%
\let\hlipl\hlkwb

\usepackage{framed}
\makeatletter
\newenvironment{kframe}{%
 \def\at@end@of@kframe{}%
 \ifinner\ifhmode%
  \def\at@end@of@kframe{\end{minipage}}%
  \begin{minipage}{\columnwidth}%
 \fi\fi%
 \def\FrameCommand##1{\hskip\@totalleftmargin \hskip-\fboxsep
 \colorbox{shadecolor}{##1}\hskip-\fboxsep
     % There is no \\@totalrightmargin, so:
     \hskip-\linewidth \hskip-\@totalleftmargin \hskip\columnwidth}%
 \MakeFramed {\advance\hsize-\width
   \@totalleftmargin\z@ \linewidth\hsize
   \@setminipage}}%
 {\par\unskip\endMakeFramed%
 \at@end@of@kframe}
\makeatother

\definecolor{shadecolor}{rgb}{.97, .97, .97}
\definecolor{messagecolor}{rgb}{0, 0, 0}
\definecolor{warningcolor}{rgb}{1, 0, 1}
\definecolor{errorcolor}{rgb}{1, 0, 0}
\newenvironment{knitrout}{}{} % an empty environment to be redefined in TeX

\usepackage{alltt}]{beamer}
\usepackage{lmodern}
\usepackage{lscape}
\usepackage{graphicx}
\usepackage[utf8]{inputenc}
\usepackage[backend=biber, bibstyle=apa, citestyle=authoryear]{biblatex}
\usepackage{floatrow}
\usepackage{booktabs}
% \usepackage{bookmark}
\usepackage{rotating}
\usepackage{dcolumn}
\usepackage{mathtools}
\usepackage[left = 2.5cm, top = 2.5cm, right = 2.5cm, bottom = 2.0cm]{geometry}
% \usepackage[hidelinks]{hyperref}
% \usepackage{mathtools}
% \usepackage{amssymb}
% \usepackage{latexsym}
% \usepackage{eurosym}
% \usepackage{xcolor}
% \usepackage{graphicx}
% \usepackage{dcolumn}
% \usepackage{floatrow}
% \usepackage[english]{babel}
% \usepackage[utf8]{inputenc}
% \usepackage{subfigure}
% \usepackage[flushleft]{threeparttable}
% \usepackage{booktabs}
% \usepackage{adjustbox}
% \usepackage{sidecap}
% \usepackage{ifthen}
% \usepackage[backend=biber, bibstyle=apa, citestyle=authoryear]{biblatex}
% \usepackage[hidelinks]{hyperref}
% \graphicspath{{./Visuals/}}
\usetheme{CambridgeUS}
\usecolortheme{dolphin}
\addbibresource{references.bib}



% Fancy fit image command with optional caption copied from https://www.patrickbaylis.com/posts/2018-10-11-beamer-resizing/
\makeatletter
\newcommand{\fitimage}[2][\@nil]{
	\begin{figure}
		\begin{adjustbox}{width=0.9\textwidth, totalheight=\textheight-2\baselineskip-2\baselineskip,keepaspectratio}
			\includegraphics{#2}
		\end{adjustbox}
		\def\tmp{#1}%
		\ifx\tmp\@nnil
		\else
		\caption{#1}
		\fi
	\end{figure}
}
\makeatother

\rmfamily
\title[Analyse der Survey-Daten von CHILDREN]{Analyse der Survey-Daten von CHILDREN for a better World e.V. 
	% \small{\textit{following Kehrig: The Cyclical Nature of the Productivity Distribution}}
	}
\author[Laura, Laura, Jonathan, Rafael und Yannick]{
Laura Huber\\
\and
Laura Jepsen\\
\and
Jonathan Kirschner\\
\and
Rafael Schütz\\
\and 
Yannick Zurl\\
\and
Studentisches Praxisprojekt zur Empirischen Wirtschaftsforschung PaRE3To\\
\and
Ludwig-Maximilians-Universität München}
\date{March 3, 2020}
\IfFileExists{upquote.sty}{\usepackage{upquote}}{}
\begin{document}
	\begin{frame}
	\maketitle
\end{frame}

\frame{\frametitle{Table of Contents}
	\tableofcontents
}

\frame{\frametitle{List of Tables}
	\listoftables
}

\frame{\frametitle{List of Figures}
	\listoffigures
}

\section{Introduction}
\subsection*{Terminology}

\frame{\frametitle{Terminology}
	\begin{itemize}
		\item{Properties of recessions:}
		\begin{itemize}
			\item{Cleansing effects: relatively unproductive firms exit in and around recessions}
			\item{Sullying effects: more firms become relatively unproductive/ unproductive firms become 
				even more unproductive in and around recessions}
		\end{itemize}
		\item{Effects on productivity dispersion:}
		\begin{itemize}
			\item{Productivity dispersion is \textit{procyclical}, when cleansing effects are at work}
			\item{Productivity dispersion is \textit{countercyclical}, when sullying effects are dominant}
		\end{itemize}
	\end{itemize}
}

\section{Summary Statistics}

\frame{\frametitle{Fundamental Dynamic}}
% latex table generated in R 3.6.2 by xtable 1.8-4 package
% Sun Mar 01 14:59:29 2020
\begin{table}[ht]
\centering
\begin{tabular}{lccccc}
  \hline
 & Year & Beneficiaries, Meals & Beneficiaries, Trips & Organizations, Meals & Organizations, Trips \\ 
  \hline
1 & 2011 & 3748.0 &  & 52 &  \\ 
  2 & 2012 & 3556.0 & 2803.0 & 51 & 44 \\ 
  3 & 2013 & 4015.0 & 2823.0 & 55 & 42 \\ 
  4 & 2014 & 4685.0 & 2752.0 & 55 & 43 \\ 
  5 & 2015 & 5857.0 & 3823.0 & 55 & 49 \\ 
  6 & 2016 & 3075.0 & 3819.0 & 59 & 48 \\ 
  7 & 2017 & 4895.0 & 4150.0 & 64 & 48 \\ 
  8 & 2018 & 5102.5 & 6911.0 & 68 & 49 \\ 
   \hline
\end{tabular}
\caption{Summary Statistics} 
\label{fundamentalDynamics}
\end{table}


% \frame{\frametitle{Illustration of concepts}
% 	\begin{figure}[h]
% 		\includegraphics[width = 5.5cm]{gdp_growth.pdf}
% 		\includegraphics[width = 5.5cm]{norm_distribution.pdf}
% 		\caption{Business Cycles and Probability Densitiy Function (\footnotesize{\textit{Source: Own Illustration}})}
% 	\end{figure}
% }
% 
% \subsection*{Relevance}
% \frame{\frametitle{Why Should We Care?}
% 	\begin{enumerate}
% 		\item[1]{Relevance for the economy \& society:}
% 		\begin{itemize}
% 			\item{Various important connections between the nature of recessions and economic welfare:
% 				\begin{itemize}
% 					\item{Short-run: Besides undesirable effects of recessions (e.g. unemployment, higher government deficits, lower 
% 						profiability), do recessions have a \textit{creative destruction}-quality as proposed 
% 						by Pre-Keynesian liquidationists?}
% 					\item{Long-run: Are recessions crucial for promoting long-run technological progress, i.e. productivity 
% 						growth?}
% 				\end{itemize}
% 			}
% 		\end{itemize}
% 		\item[2]{Economic policy implications:
% 			\begin{itemize}
% 				\item{Should fiscal and monetary policy even focus on preventing or mitigating economic recessions?}
% 				\item{If recessions have (undesirable) sullying properties, which policy instruments 
% 					could be helpful for eliminating respective causes (e.g. market imperfections, liquiditiy frictions)}
% 		\end{itemize}}
% 	\end{enumerate}
% }
% 
% \frame{\frametitle{Why Should We Care?}
% 	\begin{enumerate}    
% 		\item[3]{Academia:}
% 		\begin{itemize}
% 			\item{Assessing the behavior of micro-level variables over the business cycle is essential 
% 				for comprehending and predicting aggregate macroeconomic outcomes}
% 			\item{Which theoretical models in terms of recession properties are backed by empirical evidence?}
% 		\end{itemize}
% 	\end{enumerate}
% }
% 
% \subsection*{Literature}
% \frame{\frametitle{Related Literature}
% 	\begin{itemize}
% 		\item{Extensive (recent) literature on how firm-level-variables and their cross-sectional dispersion behave over the business cycle 
% 			(see \cite{10.1257/aer.104.4.1392}, \cite{doi:10.1111/j.0013-0427.2004.00371.x}, \cite{NBERw18245})}
% 		\item{Identification of recession properties via resource (i.e. production input) reallocation 
% 			in and around times of economic downturn (e.g. \cite{10.1111/j.1467-937X.2005.00334.x}, \cite{Mustre})}
% 		\item{Behavior of productivity dispersion over the business cycle \cite{kehrig}:}
% 		\begin{itemize}
% 			\item{Productivity dispersion is \textit{countercyclical} and particularly peaks during 
% 				or following recessions}
% 			\item{Results are driven by unproductive firms, becoming even more unproductive during recessions}
% 			\item{Evidence supports sullying properties}
% 		\end{itemize}
% 	\end{itemize}
% }
% 
% \section{Data}
% 
% \frame{\frametitle{Data Sources}
% 	\begin{itemize}
% 		\item{Primary source: COMPUSTAT Capital IQ - Fundamentals Annual in North America}
% 		\begin{itemize}
% 			\item{Provider: Standard \& Poor's (S\&P), data is primarly drawn from SEC filings}
% 			\item{Contains annual financial information on publicly traded, active and inactive firms}
% 			\item{Sample selection issues \cite{Davis}}
% 			\item{Initial sample: 1950-2018 and firms from all divisions ($n = 450.000$)}
% 		\end{itemize}
% 		\item{Additional sources:}
% 		\begin{itemize}
% 			\item{Bureau of Economic Analysis: price deflators, GDP-growth and real GDP-values, gross 
% 				output by industry-data}
% 			\item{Bureau of Labor Statistics: entry and exit rates within divisions}
% 		\end{itemize}
% 	\end{itemize}
% }
% 
% \frame{\frametitle{Variable selection and grouping levels}
% 	\begin{itemize}
% 		\item{Variables of interest:}
% 		\begin{enumerate}
% 			\item{Identification variables: GVKEY, fyear}
% 			\item{Variables for selection purposes: FIC, SIC, AQC, AT}
% 			\item{Firm-level variables for analysis: PPEGT, PPENT, SALE, EMP, COGS}
% 		\end{enumerate}
% 		\item{Data grouping into divisions and industries via SIC codes}
% 		\item{Within manufacturing: distinction between durable-goods- and non-durable-goods-producing firms}
% 	\end{itemize}
% }
% 
% \frame{\frametitle{Data grouping}
% 	\begin{figure}[h]
% 		\centering
% 		\includegraphics[width = 5.5cm]{Divisions.JPG}
% 		\includegraphics[width = 5.5cm]{Manufacturing.JPG}
% 		\caption{Data grouping levels according to SIC-codes \newline \footnotesize{\textit{Source: https://www.naics.com/sic-codes-industry-drilldown/}}}
% 	\end{figure}
% }
% 
% \frame{\frametitle{Sample Selection and Variable ``Treatments''}
% 	\begin{enumerate}
% 		\item{Restrict analysis to US-firms}
% 		\item{Main focus: manufacturing firms, supplemented by other division(s) for comparative 
% 			reasons (e.g. services)}
% 		\item{Exclude:}
% 		\begin{itemize}
% 			\item{Utility firms and financial firms}
% 			\item{Observations in which $AQC > 5\%~ of ~AT$}
% 			\item{Observations with no entries on firm-level variables}
% 			\item{Observations with $SALE > 0$ \& $EMP = 0$}
% 			\item{Observations with $PPEGT = 0$ or $PPENT = 0$}
% 			\item{Years with too little observations (per industry) to allow for statistically 
% 				reliable results}
% 		\end{itemize}
% 		\item{Denominate nominal value variables in real values by deflating}
% 	\end{enumerate}
% }
% 
% \frame{\frametitle{Sample Size}
% 	\begin{figure}[h]
% 		\centering
% 		\includegraphics[width = 5cm]{Observations1.JPG}
% 		\includegraphics[width = 5cm]{Observations2.JPG}
% 		\caption{Annual number of observations, (\textit{Source: Own calculations})}
% 	\end{figure}
% }
% 
% \frame{\frametitle{Summary Statistics}
% 	\begin{table}[h]
% 		\footnotesize
% 		\centering
% 		\caption{Descriptive statistics of production inputs and outputs}
% 		\begin{tabular}{p{2.0cm}p{1.3cm}p{1.3cm}p{1.3cm}p{1.3cm}p{1.3cm}}
% 			\hline\hline \\
% 			\textbf{Sample} &\hfill Capital &\hfill Labor &\hfill Interm. &\hfill Sale &\hfill \textbf{n} \\
% 			\hline
% 			\textbf{Manufacturing} &\hfill &\hfill &\hfill &\hfill &\hfill \textbf{107783}\\ 
% 			mean &\hfill 691.18 &\hfill 4.84 &\hfill 985.16 &\hfill 1400.99 \\
% 			$\sigma$ &\hfill 5188.13 &\hfill 20.53 &\hfill 6812.45 &\hfill 8815.76  \\
% 			\hline
% 			\textbf{Services} &\hfill &\hfill &\hfill &\hfill &\hfill \textbf{38218}\\
% 			mean &\hfill 297.95 &\hfill 5.29 &\hfill 340.46  &\hfill 580.84  \\
% 			$\sigma$ &\hfill 2162.94 &\hfill 28.37 &\hfill 1823.36 &\hfill 3563.94 \\
% 			\hline 
% 			\textbf{Durables} &\hfill &\hfill &\hfill &\hfill &\hfill \textbf{67020} \\
% 			mean &\hfill 464.25 &\hfill 4.62 &\hfill 780.05 &\hfill 1078.07 \\
% 			\hline
% 			\textbf{Non-Durables} &\hfill &\hfill &\hfill &\hfill &\hfill \textbf{40763} \\
% 			mean &\hfill 1068.12 &\hfill 5.21 &\hfill 1322.36 &\hfill 1931.90 \\
% 			\hline %need to find a double-hline!
% 			\textbf{Overall} &\hfill &\hfill &\hfill &\hfill &\hfill \textbf{146001} \\
% 			\hline\hline
% 			% Sigma-values need to denominated in smaller letters, or alternatively brackets. 
% 		\end{tabular} \\
% 		\vspace{2mm}
% 		{\tiny \textit{Note: Capital, intermediaries and sales are denominated in mio. USD, labor 
% 				is denominated in thousands of employees. All figures are rounded to two decimal digits.}}
% 		\label{tab: 1}
% 	\end{table}  
% }
% 
% \section{Dispersion Measurement}
% \frame{\frametitle{Estimating Productivity}
% 	\begin{equation}
% 	ln~y_{it} = \beta_0 + \beta_k~ln~k_{it-1} + \beta_n ~ ln~n_{it} + \beta_m ~ ln~m_{it} 
% 	+ \beta_i D_i + \beta_t D_t + \epsilon_{it}
% 	\end{equation}
% 	\begin{itemize}
% 		\small
% 		\item{$y_{it}$: Real Sales}
% 		\item{$k_{it-1}$: Constructed Capital via PIM:}
% 		\item{$n_{it}$: Number of Employees}
% 		\item{$m_{it}$: Real Cost of Sold Goods}
% 		\item{$D_i$: Firm Dummy}
% 		\item{$D_t$: Year Dummy}
% 		\item{$\epsilon_{it}$: Firm-level productivity}
% 	\end{itemize}
% 	
% }
% 
% \frame{\frametitle{Compute Productivity Dispersion}
% 	\begin{block}{\textbf{\color{red}Goal:}}
% 		\color{black} $\Rightarrow$ Compute an aggregated productivity 
% 		dispersion measure for every industry
% 	\end{block}
% 	\begin{enumerate}
% 		\item[1.]{Adjustment: control for industry specific productivity growth trend \newline
% 			$\epsilon_{ijt} = g_{j}t + z_{ijt}$}
% 		\begin{itemize}
% 			\item{$\epsilon_{ijt}$: Productivity}
% 			\item{$g_j$: Industry-specific growth trend}
% 			\item{$z_{ijt}$: Productivity dispersion}
% 		\end{itemize}
% 		\item[2.]{Adjustment: decenter and scale to construct cross-sectional dispersion}
% 		\item[3.]{Adjustment: control for a long-term trend}
% 	\end{enumerate}
% }
% 
% \frame{\frametitle{Dispersion Measure}
% 	\begin{figure}[h]
% 		\centering
% 		\includegraphics[width = 12cm]{Dispersion.JPG}
% 		\caption{Illustration of adjustment steps (\footnotesize{\textit{Source: Kehrig (2015)}})}
% 	\end{figure}
% }
% 
% \frame{\frametitle{The Distribution of Normalized Productivity Dispersion}
% 	\begin{figure}[h]
% 		\centering
% 		\includegraphics[height = 4.5cm, width = 5.5cm]{28_1998.pdf}
% 		\includegraphics[height = 4.5cm, width = 5.5cm]{34_2003.pdf}
% 		\caption{Exemplary normalized productivity dispersion for two industries, (\textit{Source: Own illustration})}
% 	\end{figure}
% }
% 
% \frame{\frametitle{The Normalized Productivity Dispersion}
% 	\begin{minipage}[c]{0.6\textwidth}
% 		\begin{figure}[h]
% 			\centering
% 			\includegraphics[width = 7cm]{normalized_productvitiy.pdf}
% 			\caption{Normalized productivity over years}
% 		\end{figure}
% 	\end{minipage}
% 	\begin{minipage}[b]{0.2\textwidth}
% 		\begin{flalign*}
% 		Dispersion_t &= \\
% 		Median_t[Std_{jt}&(\frac{z_{ijt} - \bar{z_j}}{\sigma_j})]
% 		\end{flalign*}
% 	\end{minipage}
% }
% \frame{\frametitle{Growth-Trends in Productivity-Dispersion}
% 	\begin{figure}[h]
% 		\centering
% 		\includegraphics[width = 8.5cm]{long_final.pdf}
% 		\caption{Actual dispersion behavior and trends over time, (\textit{Source: Own Illustration})}
% 	\end{figure}
% }
% 
% \frame{\frametitle{First Estimations}
% 	\begin{table}[htb]
% 		\centering
% 		\caption{Results of OLS Regression}\vspace{1ex}
% 		\begin{tabular}{@{}lcccc@{}}
% 			\toprule[2pt]
% 			{\bf ln\_sale} & Coefficient & Robust St. Err. & t-Statistics & p-value \\
% 			\hline
% 			\bf ln\_capital & 0.4300*** & 0.0097 & 44.12 & 0.000\\
% 			\bf ln\_emp &0.4801*** & 0.0099 & 48.20 & 0.000\\
% 			\bf ln\_cog &0.1035*** & 0.0064 & 16.13 & 0.000\\
% 			\hline 
% 			\midrule
% 			\multicolumn{5}{c}{R-squared=0.9681 Adj.R-squared=0.969 RMSE=0.4686}\\
% 			\bottomrule[2pt]
% 		\end{tabular}
% 	\end{table}
% }
% 
% \frame{\frametitle{First Estimations}
% 	\begin{figure}[h]
% 		\centering
% 		\includegraphics[width = 9cm]{fixed_final.pdf}
% 		\caption{Annual time-FE-coefficients and GDP-growth (\textit{Source: Own Illustration})}
% 	\end{figure}
% }
% 
% \section{Empirical Analysis}
% \frame{\frametitle{Duables and Non-Durables Within Manufacturing}
% 	\begin{figure}[h]
% 		\centering
% 		\includegraphics[width = 8cm]{durable_final.pdf}
% 		\caption{Dispersion of productivity levels within durables and non-durables, (\textit{Source: Own Illustration})}
% 	\end{figure}
% }
% 
% \frame{\frametitle{AutoscaledLineDispTot}
% 	\fitimage{LineDispTot.pdf}
% }
% 
% \frame{\frametitle{}
% 	\begin{figure}[h]
% 		\centering
% 		\includegraphics[]{LineDispTot.pdf}
% 		\caption{(\textit{Source: Own Illustration})}
% 	\end{figure}
% }
% 
% \frame{\frametitle{}
% 	\begin{figure}[h]
% 		\centering
% 		\includegraphics[width = 9cm]{ScatDispTot.pdf}
% 		\caption{(\textit{Source: Own Illustration})}
% 	\end{figure}
% }
% 
% \frame{\frametitle{}
% 	\begin{table}[ht]
% 		\centering
% 		\begin{adjustbox}{width=\textwidth, totalheight=\textheight-2\baselineskip, keepaspectratio}
% 			\begin{tabular}{ccccccc}
% 				\hline
% 				Lead in years & All & p-value & Non-Durables & p-value & Durables & p-value \\ 
% 				\hline
% 				-2 & 0.195 & 0.12 & -0.144 & 0.30 & 0.030 & 0.83 \\ 
% 				1 & 0.038 & 0.76 & 0.101 & 0.47 & 0.186 & 0.18 \\ 
% 				0 & -0.045 & 0.71 & 0.132 & 0.34 & -0.184 & 0.18 \\ 
% 				1 & 0.038 & 0.76 & 0.101 & 0.47 & 0.186 & 0.18 \\ 
% 				2 & -0.160 & 0.20 & 0.056 & 0.69 & 0.297 & 0.03 \\ 
% 				\hline
% 			\end{tabular}
% 		\end{adjustbox}
% 	\end{table}
% }
% 
% \frame{\frametitle{}
% 	\begin{figure}[h]
% 		\centering
% 		\includegraphics[width = 9cm]{PlotQuantilesLow.pdf}
% 		\caption{(\textit{Source: Own Illustration})}
% 	\end{figure}
% }
% 
% \frame{\frametitle{}
% 	\begin{figure}[h]
% 		\centering
% 		\includegraphics[width = 9cm]{PlotQuantilesHigh.pdf}
% 		\caption{(\textit{Source: Own Illustration})}
% 	\end{figure}
% }
% 
% \frame{\frametitle{Dynamics in the Manufacturing Division}
% 	\begin{figure}[h]
% 		\centering
% 		\includegraphics[width=8cm]{quantile_m_final.pdf}
% 		\caption{Productivity changes in different manufacturing quantiles, (\textit{Source: Own illustration})}
% 	\end{figure}
% }
% 
% \section{Discussion}
% \frame{\frametitle{}
% 	\begin{table}[ht]
% 		\centering
% 		\begin{adjustbox}{width=\textwidth, totalheight=\textheight-2\baselineskip, keepaspectratio}
% 			\begin{tabular}{lcccccc}
% 				\hline
% 				& All & p-value & Non-durables & p-value & Durables & p-value \\ 
% 				\hline
% 				Standard Deviation & -0.045 & 0.71 & 0.132 & 0.34 & -0.184 & 0.18 \\ 
% 				Variance & -0.045 & 0.71 & 0.132 & 0.34 & -0.184 & 0.18 \\ 
% 				Inter-quartile range & 0.005 & 0.97 & 0.114 & 0.41 & -0.140 & 0.31 \\ 
% 				Median absolute deviation & -0.023 & 0.86 & 0.135 & 0.33 & -0.091 & 0.51 \\ 
% 				\hline
% 			\end{tabular}
% 		\end{adjustbox}
% 	\end{table}
% }
% 
% \frame{\frametitle{}
% 	\begin{table}[ht]
% 		\centering
% 		\begin{adjustbox}{{width=\textwidth, totalheight=\textheight-2\baselineskip, keepaspectratio}
% 				\begin{tabular}{lcccccc}
% 					\hline
% 					& All & p-value & Non-durables & p-value & Durables & p-value \\ 
% 					\hline
% 					Production - HP filtered ($\lambda = 100$) &  &  & 0.132 & 0.34 & -0.184 & 0.18 \\ 
% 					Production - HP filtered ($\lambda = 6.25$) &  &  & 0.150 & 0.28 & -0.322 & 0.02 \\ 
% 					Production growth rate &  &  & 0.032 & 0.82 & -0.343 & 0.01 \\ 
% 					GDP - HP filtered ($\lambda = 100$) & -0.045 & 0.71 & 0.058 & 0.64 & -0.100 & 0.42 \\ 
% 					GDP - HP filtered ($\lambda = 6.25$) & -0.121 & 0.33 & 0.109 & 0.38 & -0.129 & 0.29 \\ 
% 					GDP growth rate & -0.043 & 0.73 & -0.041 & 0.74 & 0.018 & 0.89 \\ 
% 					\hline
% 				\end{tabular}
% 			\end{adjustbox}
% 		\end{table}
% 	}
% 	
% 	\section{Conclusion}
% 	\frame{\frametitle{Conclusion}
% 		\begin{itemize}
% 			\item{\textbf{Main result:}}
% 			\begin{itemize}
% 				\item{No clear general cyclical or countercyclical pattern}
% 				\item{But: Productivitiy dispersion is countercyclical for durables}
% 				\item{Recessions do have (small) sullying effects, in particular in the manufacturing division}
% 			\end{itemize}
% 			\item{\textbf{Where to go from here?}}
% 			\begin{itemize}
% 				\item{Replicate findings with quarterly data}
% 				\item{Did policy interventions in the analyzed NBER-recessions already have an impact (potential OVB)?}
% 				\item{Identify causal link for sullying effecs: Why do unproductive firms remain in business (potential frictions)?}
% 			\end{itemize}
% 		\end{itemize}
% 	}
	\frame[allowframebreaks]{\frametitle{References}
		\tiny\printbibliography
	}
	
\end{document}


