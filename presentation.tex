<<<<<<< HEAD
% in Terminal enter
=======
>>>>>>> 49d207dd3cfa83951ab0a285bb08a0f96c513c41
% Rscript -e "library(knitr); knit('presentation.Rnw')"
% pdflatex presentation
% biber presentation
% pdflatex presentation
% pdflatex presentation

<<<<<<< HEAD
% https://github.com/yihui/knitr/blob/master/inst/examples/knitr-beamer.Rnw



\documentclass{beamer}\usepackage[]{graphicx}\usepackage[]{color}
=======
\documentclass[hyperref={pdfpagelabels=false}\usepackage[]{graphicx}\usepackage[]{color}
>>>>>>> 49d207dd3cfa83951ab0a285bb08a0f96c513c41
% maxwidth is the original width if it is less than linewidth
% otherwise use linewidth (to make sure the graphics do not exceed the margin)
\makeatletter
\def\maxwidth{ %
  \ifdim\Gin@nat@width>\linewidth
    \linewidth
  \else
    \Gin@nat@width
  \fi
}
\makeatother

\definecolor{fgcolor}{rgb}{0.345, 0.345, 0.345}
\newcommand{\hlnum}[1]{\textcolor[rgb]{0.686,0.059,0.569}{#1}}%
\newcommand{\hlstr}[1]{\textcolor[rgb]{0.192,0.494,0.8}{#1}}%
\newcommand{\hlcom}[1]{\textcolor[rgb]{0.678,0.584,0.686}{\textit{#1}}}%
\newcommand{\hlopt}[1]{\textcolor[rgb]{0,0,0}{#1}}%
\newcommand{\hlstd}[1]{\textcolor[rgb]{0.345,0.345,0.345}{#1}}%
\newcommand{\hlkwa}[1]{\textcolor[rgb]{0.161,0.373,0.58}{\textbf{#1}}}%
\newcommand{\hlkwb}[1]{\textcolor[rgb]{0.69,0.353,0.396}{#1}}%
\newcommand{\hlkwc}[1]{\textcolor[rgb]{0.333,0.667,0.333}{#1}}%
\newcommand{\hlkwd}[1]{\textcolor[rgb]{0.737,0.353,0.396}{\textbf{#1}}}%
\let\hlipl\hlkwb

\usepackage{framed}
\makeatletter
\newenvironment{kframe}{%
 \def\at@end@of@kframe{}%
 \ifinner\ifhmode%
  \def\at@end@of@kframe{\end{minipage}}%
  \begin{minipage}{\columnwidth}%
 \fi\fi%
 \def\FrameCommand##1{\hskip\@totalleftmargin \hskip-\fboxsep
 \colorbox{shadecolor}{##1}\hskip-\fboxsep
     % There is no \\@totalrightmargin, so:
     \hskip-\linewidth \hskip-\@totalleftmargin \hskip\columnwidth}%
 \MakeFramed {\advance\hsize-\width
   \@totalleftmargin\z@ \linewidth\hsize
   \@setminipage}}%
 {\par\unskip\endMakeFramed%
 \at@end@of@kframe}
\makeatother

\definecolor{shadecolor}{rgb}{.97, .97, .97}
\definecolor{messagecolor}{rgb}{0, 0, 0}
\definecolor{warningcolor}{rgb}{1, 0, 1}
\definecolor{errorcolor}{rgb}{1, 0, 0}
\newenvironment{knitrout}{}{} % an empty environment to be redefined in TeX

<<<<<<< HEAD
\usepackage{alltt}
% \usepackage{lmodern}
% \usepackage{lscape}
% \usepackage{graphicx}
% \usepackage[utf8]{inputenc}
% \usepackage[backend=biber, bibstyle=apa, citestyle=authoryear]{biblatex}
% \usepackage{floatrow}
% \usepackage{booktabs}
% \usepackage{rotating}
% \usepackage{dcolumn}
% \usepackage{mathtools}
% \usepackage[left = 2.5cm, top = 2.5cm, right = 2.5cm, bottom = 2.0cm]{geometry}
=======
\usepackage{alltt}]{beamer}
\usepackage{lmodern}
\usepackage{lscape}
\usepackage{graphicx}
\usepackage[utf8]{inputenc}
\usepackage[backend=biber, bibstyle=apa, citestyle=authoryear]{biblatex}
\usepackage{floatrow}
\usepackage{booktabs}
% \usepackage{bookmark}
\usepackage{rotating}
\usepackage{dcolumn}
\usepackage{mathtools}
\usepackage[left = 2.5cm, top = 2.5cm, right = 2.5cm, bottom = 2.0cm]{geometry}
>>>>>>> 49d207dd3cfa83951ab0a285bb08a0f96c513c41
% \usepackage[hidelinks]{hyperref}
% \usepackage{mathtools}
% \usepackage{amssymb}
% \usepackage{latexsym}
% \usepackage{eurosym}
% \usepackage{xcolor}
% \usepackage{graphicx}
% \usepackage{dcolumn}
% \usepackage{floatrow}
% \usepackage[english]{babel}
% \usepackage[utf8]{inputenc}
% \usepackage{subfigure}
% \usepackage[flushleft]{threeparttable}
% \usepackage{booktabs}
% \usepackage{adjustbox}
% \usepackage{sidecap}
% \usepackage{ifthen}
% \usepackage[backend=biber, bibstyle=apa, citestyle=authoryear]{biblatex}
% \usepackage[hidelinks]{hyperref}
% \graphicspath{{./Visuals/}}
% \setcounter{secnumdepth}{3}
% \setcounter{tocdepth}{3}
% \usepackage{url}
% \ifx\hypersetup\undefined
%   \AtBeginDocument{%
%     \hypersetup{unicode=true,pdfusetitle,
%  bookmarks=true,bookmarksnumbered=false,bookmarksopen=false,
%  breaklinks=false,pdfborder={0 0 0},pdfborderstyle={},backref=false,colorlinks=false}
%   }
% \else
%   \hypersetup{unicode=true,pdfusetitle,
%  bookmarks=true,bookmarksnumbered=false,bookmarksopen=false,
%  breaklinks=false,pdfborder={0 0 0},pdfborderstyle={},backref=false,colorlinks=false}
% \fi
% \usepackage{breakurl}
% 
% \makeatletter
%
% Fancy fit image command with optional caption copied from https://www.patrickbaylis.com/posts/2018-10-11-beamer-resizing/
% \makeatletter
% \newcommand{\fitimage}[2][\@nil]{
% 	\begin{figure}
% 		\begin{adjustbox}{width=0.9\textwidth, totalheight=\textheight-2\baselineskip-2\baselineskip,keepaspectratio}
% 			\includegraphics{#2}
% 		\end{adjustbox}
% 		\def\tmp{#1}%
% 		\ifx\tmp\@nnil
% 		\else
% 		\caption{#1}
% 		\fi
% 	\end{figure}
% }
% \makeatother
% 
% \rmfamily

\usetheme{CambridgeUS}
% \usecolortheme{dolphin}
% \addbibresource{references.bib}
%
\IfFileExists{upquote.sty}{\usepackage{upquote}}{}
\begin{document}



<<<<<<< HEAD

\title[Analyse der Survey-Daten von CHILDREN]{Analyse der Survey-Daten von CHILDREN for a better World e.V.
=======
% Fancy fit image command with optional caption copied from https://www.patrickbaylis.com/posts/2018-10-11-beamer-resizing/
\makeatletter
\newcommand{\fitimage}[2][\@nil]{
	\begin{figure}
		\begin{adjustbox}{width=0.9\textwidth, totalheight=\textheight-2\baselineskip-2\baselineskip,keepaspectratio}
			\includegraphics{#2}
		\end{adjustbox}
		\def\tmp{#1}%
		\ifx\tmp\@nnil
		\else
		\caption{#1}
		\fi
	\end{figure}
}
\makeatother

\rmfamily
\title[Analyse der Survey-Daten von CHILDREN]{Analyse der Survey-Daten von CHILDREN for a better World e.V. 
	% \small{\textit{following Kehrig: The Cyclical Nature of the Productivity Distribution}}
>>>>>>> 49d207dd3cfa83951ab0a285bb08a0f96c513c41
	}
\author[Laura, Laura, Jonathan, Rafael und Yannick]{
Laura Huber\\
\and
Laura Jepsen\\
\and
Jonathan Kirschner\\
\and
Rafael Schütz\\
\and
Yannick Zurl\\
\and
Studentisches Praxisprojekt zur Empirischen Wirtschaftsforschung PaRE3To\\
\and
Ludwig-Maximilians-Universität München}
\date{March 3, 2020}


\begin{frame}
	\maketitle
\end{frame}

\frame{\frametitle{Table of Contents}
	\tableofcontents
}

\frame{\frametitle{List of Tables}
	\listoftables
}

\frame{\frametitle{List of Figures}
	\listoffigures
}

<<<<<<< HEAD
\section{Summary Statistics}

\begin{frame}[fragile]
% latex table generated in R 3.6.2 by xtable 1.8-4 package
% Sat Feb 29 22:45:36 2020
=======
\section{Introduction}
\subsection*{Terminology}

\frame{\frametitle{Terminology}
	\begin{itemize}
		\item{Properties of recessions:}
		\begin{itemize}
			\item{Cleansing effects: relatively unproductive firms exit in and around recessions}
			\item{Sullying effects: more firms become relatively unproductive/ unproductive firms become 
				even more unproductive in and around recessions}
		\end{itemize}
		\item{Effects on productivity dispersion:}
		\begin{itemize}
			\item{Productivity dispersion is \textit{procyclical}, when cleansing effects are at work}
			\item{Productivity dispersion is \textit{countercyclical}, when sullying effects are dominant}
		\end{itemize}
	\end{itemize}
}

\section{Summary Statistics}

\frame{\frametitle{Fundamental Dynamic}}
<<<<<<< HEAD
% latex table generated in R 3.6.2 by xtable 1.8-4 package
% Sat Feb 29 17:52:27 2020
=======
% latex table generated in R 3.5.1 by xtable 1.8-4 package
% Sat Feb 29 17:55:38 2020
>>>>>>> b24662a116b8b0301ff26431155b33d6392d6dc5
>>>>>>> 49d207dd3cfa83951ab0a285bb08a0f96c513c41
\begin{table}[ht]
\centering
\scalebox{0.75}{
\begin{tabular}{lccccc}
  \hline
 & Year & Beneficiaries, Meals & Beneficiaries, Trips & Organizations, Meals & Organizations, Trips \\ 
  \hline
1 & 2011 & 3748.0 &  & 52 &  \\ 
  2 & 2012 & 3556.0 & 2803.0 & 51 & 44 \\ 
  3 & 2013 & 4015.0 & 2823.0 & 55 & 42 \\ 
  4 & 2014 & 4685.0 & 2752.0 & 55 & 43 \\ 
  5 & 2015 & 5857.0 & 3823.0 & 55 & 49 \\ 
  6 & 2016 & 3075.0 & 3819.0 & 59 & 48 \\ 
  7 & 2017 & 4895.0 & 4150.0 & 64 & 48 \\ 
  8 & 2018 & 5102.5 & 6911.0 & 68 & 49 \\ 
   \hline
\end{tabular}
}
\caption{Summary Statistics} 
\label{fundamentalDynamics}
\end{table}

\end{frame}

<<<<<<< HEAD
% \begin
% \frame[allowframebreaks]{\frametitle{References}
% 		\tiny
% 		\printbibliography
=======
% \frame{\frametitle{Illustration of concepts}
% 	\begin{figure}[h]
% 		\includegraphics[width = 5.5cm]{gdp_growth.pdf}
% 		\includegraphics[width = 5.5cm]{norm_distribution.pdf}
% 		\caption{Business Cycles and Probability Densitiy Function (\footnotesize{\textit{Source: Own Illustration}})}
% 	\end{figure}
% }
% 
% \subsection*{Relevance}
% \frame{\frametitle{Why Should We Care?}
% 	\begin{enumerate}
% 		\item[1]{Relevance for the economy \& society:}
% 		\begin{itemize}
% 			\item{Various important connections between the nature of recessions and economic welfare:
% 				\begin{itemize}
% 					\item{Short-run: Besides undesirable effects of recessions (e.g. unemployment, higher government deficits, lower 
% 						profiability), do recessions have a \textit{creative destruction}-quality as proposed 
% 						by Pre-Keynesian liquidationists?}
% 					\item{Long-run: Are recessions crucial for promoting long-run technological progress, i.e. productivity 
% 						growth?}
% 				\end{itemize}
% 			}
% 		\end{itemize}
% 		\item[2]{Economic policy implications:
% 			\begin{itemize}
% 				\item{Should fiscal and monetary policy even focus on preventing or mitigating economic recessions?}
% 				\item{If recessions have (undesirable) sullying properties, which policy instruments 
% 					could be helpful for eliminating respective causes (e.g. market imperfections, liquiditiy frictions)}
% 		\end{itemize}}
% 	\end{enumerate}
% }
% 
% \frame{\frametitle{Why Should We Care?}
% 	\begin{enumerate}    
% 		\item[3]{Academia:}
% 		\begin{itemize}
% 			\item{Assessing the behavior of micro-level variables over the business cycle is essential 
% 				for comprehending and predicting aggregate macroeconomic outcomes}
% 			\item{Which theoretical models in terms of recession properties are backed by empirical evidence?}
% 		\end{itemize}
% 	\end{enumerate}
>>>>>>> 49d207dd3cfa83951ab0a285bb08a0f96c513c41
% }
	
\end{document}


